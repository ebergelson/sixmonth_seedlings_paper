\documentclass[]{article}
\usepackage{lmodern}
\usepackage{amssymb,amsmath}
\usepackage{ifxetex,ifluatex}
\usepackage{fixltx2e} % provides \textsubscript
\ifnum 0\ifxetex 1\fi\ifluatex 1\fi=0 % if pdftex
  \usepackage[T1]{fontenc}
  \usepackage[utf8]{inputenc}
\else % if luatex or xelatex
  \ifxetex
    \usepackage{mathspec}
  \else
    \usepackage{fontspec}
  \fi
  \defaultfontfeatures{Ligatures=TeX,Scale=MatchLowercase}
\fi
% use upquote if available, for straight quotes in verbatim environments
\IfFileExists{upquote.sty}{\usepackage{upquote}}{}
% use microtype if available
\IfFileExists{microtype.sty}{%
\usepackage{microtype}
\UseMicrotypeSet[protrusion]{basicmath} % disable protrusion for tt fonts
}{}
\usepackage[margin=1in]{geometry}
\usepackage{hyperref}
\hypersetup{unicode=true,
            pdftitle={SI For The Nature and Origins of the Lexicon in Six-month-olds},
            pdfauthor={Elika Bergelson \& Richard Aslin},
            pdfborder={0 0 0},
            breaklinks=true}
\urlstyle{same}  % don't use monospace font for urls
\usepackage{graphicx,grffile}
\makeatletter
\def\maxwidth{\ifdim\Gin@nat@width>\linewidth\linewidth\else\Gin@nat@width\fi}
\def\maxheight{\ifdim\Gin@nat@height>\textheight\textheight\else\Gin@nat@height\fi}
\makeatother
% Scale images if necessary, so that they will not overflow the page
% margins by default, and it is still possible to overwrite the defaults
% using explicit options in \includegraphics[width, height, ...]{}
\setkeys{Gin}{width=\maxwidth,height=\maxheight,keepaspectratio}
\IfFileExists{parskip.sty}{%
\usepackage{parskip}
}{% else
\setlength{\parindent}{0pt}
\setlength{\parskip}{6pt plus 2pt minus 1pt}
}
\setlength{\emergencystretch}{3em}  % prevent overfull lines
\providecommand{\tightlist}{%
  \setlength{\itemsep}{0pt}\setlength{\parskip}{0pt}}
\setcounter{secnumdepth}{0}
% Redefines (sub)paragraphs to behave more like sections
\ifx\paragraph\undefined\else
\let\oldparagraph\paragraph
\renewcommand{\paragraph}[1]{\oldparagraph{#1}\mbox{}}
\fi
\ifx\subparagraph\undefined\else
\let\oldsubparagraph\subparagraph
\renewcommand{\subparagraph}[1]{\oldsubparagraph{#1}\mbox{}}
\fi

%%% Use protect on footnotes to avoid problems with footnotes in titles
\let\rmarkdownfootnote\footnote%
\def\footnote{\protect\rmarkdownfootnote}

%%% Change title format to be more compact
\usepackage{titling}

% Create subtitle command for use in maketitle
\newcommand{\subtitle}[1]{
  \posttitle{
    \begin{center}\large#1\end{center}
    }
}

\setlength{\droptitle}{-2em}
  \title{SI For ``The Nature and Origins of the Lexicon in Six-month-olds''}
  \pretitle{\vspace{\droptitle}\centering\huge}
  \posttitle{\par}
  \author{Elika Bergelson \& Richard Aslin}
  \preauthor{\centering\large\emph}
  \postauthor{\par}
  \date{}
  \predate{}\postdate{}

\usepackage{booktabs}
\usepackage{longtable}
\usepackage{array}
\usepackage{multirow}
\usepackage[table]{xcolor}
\usepackage{wrapfig}
\usepackage{float}
\usepackage{colortbl}
\usepackage{pdflscape}
\usepackage{tabu}
\usepackage{threeparttable}
\setlength\parindent{24pt}

\begin{document}
\maketitle

\section{Supporting Methods and
Materials}\label{supporting-methods-and-materials}

\subsection{Eyetracking Target Word Selection and Relatedness
Quantification}\label{eyetracking-target-word-selection-and-relatedness-quantification}

The 16 target words tested in the eyetracking experiment were selected
as high-frequency concrete nouns commonly heard by infants. These words
occurred on average 613 times (R: 46-2197), from 14/16 mothers (R: 6-16)
in the Brent Corpus (an audio corpus of mothers and 9--15-month-olds)
(31). Many of the words were also used in previous word comprehension
studies of 6--9-month-olds (1-4).

To quantify the relatedness among the `related' and `unrelated'
item-pairs used in the eyetracking experiment, we used a semantic
network model (word2vec) over the North American English portion of
CHILDES (37), and computed the similarity (i.e.~cosine between two word
vectors) for each unrelated and related pair. The unrelated pairs had
significantly lower similarity (\emph{M} = 0.18) than the related pairs
(0.47; T(12.19) = 0.009) by Welch two-sample t-test. Code for this
analysis is available on github:
\url{https://github.com/SeedlingsBabylab/w2v_cosines/}.

\subsection{Home Recording Data Processing and
Sharing}\label{home-recording-data-processing-and-sharing}

For the hourlong home video-recordings, the head-camera and camcorder
feeds were merged into a single stream using Sony Vegas and exported as
.mp4 files for annotation in Datavyu. The daylong home audio recordings
were processed using LENAs proprietary algorithm (which provides
machine-generated utterance-segmentation); the raw audio-file (.wav) and
LENA output were exported for further processing. We converted the LENA
output into a CLAN-compatible file (37), for annotation. This provided a
`skeleton' with each row's time-stamp corresponding to a LENA
``utterance''. We then used Audacity's Silence Finder algorithm to
demark long periods of silence (corresponding to naptime or silent
car-rides) in the annotation files. Omitting these silent stretches left
7.25 to 16.00 hours per file (M = 11.17), which were then manually
annotated as described in the main manuscript.

Families completed an audio-video release form in which they could elect
for their recordings to be shared at several levels: sharing with the
lab only, sharing with other authorized researchers (e.g.~HomeBank and
Databrary), and sharing short excerpts for demonstration purposes in
publications and/or research talks. Families were also informed that
they could elect not to share a section of any recording, could stop the
recording for any reason, and that they should inform anyone who had
more than incidental contact with the child that they were being
recorded, and obtain their permission. Each release form was collected
after the audio and video recording for a given family were completed.
\renewcommand{\tablename}{Table S\hskip-\the\fontdimen2\font\space }
\begin{table}

\caption{\label{tab:Supplementary Table 1}Questionnaire Results. a:infants who had begun hands-and-knees crawling; b:infants who were exclusively breast-fed; c:infants who are not yet babbling. The `Missing' column reflects how many infants in the lab-and-home (LH) and lab-only (LO) sample did not fill out the questionnaire.}
\centering
\resizebox{\linewidth}{!}{\begin{tabular}[t]{l|l|l|l|l}
\hline
Measure & Lab-and-Home & Lab-Only & All Infants & Missing\\
\hline
MCDI: Tested Words & M=1.52 (2.51), R: 0-12, mode=0 & M=3.46(4.63), R: 0-15, mode=0 & M=1.96(3.18), R: 0-15, mode=0 & 1 LH\\
\hline
MCDI: All Words & M=9.43 (16.57), R: 0-75, mode=0 & M=23.85(43.41), R: 0-162, mode=0 & M=12.72(25.53), R: 0-162, mode=0 & 1 LH\\
\hline
Word Exposure & M=3.92(0.67), R: 2.09-4.93, mode=4 & M=3.92(0.6), R: 2.62-5, mode=4 & M=3.92(0.64), R: 2.21-4.95, mode=4 & 0 LH, 1 LO\\
\hline
Motor Status (a) & 2.27\% & 0\% & 1.79\% & 1 LH, 1 LO\\
\hline
Feeding Status (b) & 31.03\% & 16.67\% & 28.57\% & 15 LH, 8 LO\\
\hline
Production Status (c) & 75\% & 58.33\% & 71.43\% & 0 LH, 2 LO\\
\hline
\end{tabular}}
\end{table}

\subsection{Additional Questionnaire Description and
Sub-Analysis}\label{additional-questionnaire-description-and-sub-analysis}

As summarized in the main manuscript, parents completed a series of
questionnaires about their infants. They completed the MacArthur-Bates
Communicative Development Inventory (MCDI) Words \& Gestures Form (38),
and two motor surveys: the gross-motor section of the EMQ (39) and four
locomotion questions (40). Parents were also orally asked our Word
Exposure Survey, which asked how often they believed their child heard
our test words on a 5-point scale (1=``Never'', 5=``Several times a
day''). Parents also described infants' vocalizations, to ascertain
whether canonical babbling had begun, and were asked whether their child
was breast-fed or bottle-fed (since `milk' and `bottle' are two of our
tested words).

Parents generally did not believe that their infant understood very many
words: the modal number of words infants were reported to understand on
the MCDI is 0, with one notable outlier of 162 (see Table S1). This
pattern of MCDI results vis-a-vis in-lab data was also found in
Bergelson \& Swingley (36). We believe this reflects the inherent
difficulty in determining whether young infants, who lack overt
behavioral cues like pointing or saying words, understand the words
around them. Thus, the in-lab results we report underscore the utility
of eyetracking measures with infants of this age; subtle eye movements
may provide researchers with a way to measure comprehension that
caretakers themselves may miss or find hard to assess.

The questionnaire data, in principle, open up the possibility of further
sub-analyses. Given that the majority of infants (in our relatively
small sample of 51 six-month-olds) were not yet crawling or babbling, we
did not conduct further analyses of these abilities in relation to
in-lab or home language data, though plan to do so with data from these
children at later timepoints. For the feeding data, given that
\textasciitilde{}30\% of infants were exclusively breast-fed, we
reanalyzed the in-lab data, excluding the item ``milk'' for these
children, under the rationale that our image of milk may not have
comported with the majority of their milk experiences. All patterns of
significance reported in the main text remained the same. This may
reflect that infants have experiences with milk other than during their
own feeding (e.g.~seeing older children and adults drinking milk), or
that infants saw breast-milk in bottles (i.e.~parents may report the
child is `breast-fed exclusively' in reference to the source of the milk
rather than the delivery method.) It may also reflect that removing one
item does not change the robust patterns across conditions we report
here, especially given that nearly half of infants did not report
feeding survey results, as this questionnaire was added a few months
after data collection began (see Table S1).

\subsection{Data Exclusion}\label{data-exclusion}

As stated in the main manuscript, we opted to exclude data at the trial
level, rather than the infant level, in order to retain infants in the
home and lab analyses. Given our relatively large number of trials per
infant (n=32), we were able to proceed with analysis using trials of
comparable and sufficient data quality. Noting that practices vary
across labs and studies, we provide here the exclusion rates from
language comprehension studies with young infants, to help situate our
trial exclusion rate. In two studies with six-month-olds, Tincoff and
Jusczyk (4,5) presented infants with a single test trial, and excluded
20-33\% of the sample for failure to complete the trial due to
inattention or fussiness. Parise and Csibra (3) excluded 43\% of
nine-month-olds in a word comprehension EEG study. In the present
eyetracking study, we retain approximately half of the trials, from most
participants. Given that the lab-and-home sample (n=44) was part of a
longitudinal set of recordings and in-lab studies, ongoing research with
these same infants over time will allow us to assess whether fussiness
and inattentive behavior at six months is predictive of subsequent
behavior in eye-tracking experiments, and in measures of language
development.

\begin{figure}
\includegraphics[width=.8\linewidth]{Bergelson_Aslin_pnas_som_manualrefs_files/figure-latex/f-itemmeans-1} \def\figurename{Figure S\hskip-\the\fontdimen2\font\space }\caption{\label{fig:f-itemmeans}Item-level performance, across subjects, in each condition. Bars are ordered by cross-condition average for each item}\label{fig:f-itemmeans}
\end{figure}

\begin{figure}
\includegraphics[width=.8\linewidth]{Bergelson_Aslin_pnas_som_manualrefs_files/figure-latex/f-timecourse-1} \def\figurename{Figure S\hskip-\the\fontdimen2\font\space }\caption{\label{fig:f-timecourse}Timecourse of infant gaze by trial-type. This figure shows the proportion of fixation to the target image over time, for each trial-type (blue = unrelated; grey = related). We smooth over each 20ms bin, averaging over subjects and trials, and add 95\% bootstrapped CIs. Black vertical lines demark the target window of analysis (367-5000ms.) The baseline window is all looking before target onset, i.e. time <0}\label{fig:f-timecourse}
\end{figure}

\section{Supplementary Figures}\label{supplementary-figures}

Figures S1 and S2 supplement the analyses provided in the main
manuscript.

\section{Supplementary Multi-Media}\label{supplementary-multi-media}

The clips linked in the Supplementary Online Materials show three sample
audio clips, and two sample video clips from the corpus. Parents
provided permission for these to be shared with researchers. As
explained in the manuscript, each object word was tagged for
utterance-type, talker, and object co-presence. We provide clips with a
range of object co-presence for clarity on how this was determined in
our audio and video files.

\textbf{Audio Clip 1 (audio\_clip1\_lowobjectpresence.mp3)}: in this
clip, the infant's father discusses his missing phone with his
six-month-old. This is an example of low object co-presence, since the
phone is not present as the father discusses it.

\textbf{Audio Clip 2 (audio\_clip2\_highobjecpresence.mp3)}: in this
clip, the infant's mother and father are reading a book to their child.
We hear the pages being turned, suggesting the book and the items it
describes are present and attended to as they are mentioned.

\textbf{Audio Clip 3 (audio\_clip3\_mixedobjectpresence.mp3)}: in this
clip, the mother is nursing, and mentions a future eating-at-the-beach
event. We can infer from context that the babyfood and sand discussed
are not co-present in the scene. In contrast, the sweet potato she says
is in the child's eye was tagged as co-present.

\textbf{Video Clip 1 (video\_clip1\_lowobjecpresence.mov)}: in this
clip, the father is singing and dancing with his child. None of the
words in the song (e.g. ``chickie'', ``egg'') are co-present in the
scene.

\textbf{Video Clip 2 (video\_clip2\_highobjectpresence.mov)}: in this
clip, the mother is reading a picturebook to her child. We see from his
head-mounted cameras and the camcorder that the images in the book go
with the words that the mother is saying, and that the child is
attending to them, i.e.~they are `co-present'.


\end{document}
